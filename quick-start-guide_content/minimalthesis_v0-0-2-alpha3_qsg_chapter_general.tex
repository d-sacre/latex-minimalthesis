\chapter{General Information}
	\section{Default Settings \& User Choice}
		\subsection{Base Class \& Basic Layout}
			Per default, \verb|minimalthesis| uses the \verb|scrreprt| koma document class with a two sided layout (binding correction: \SI{1}{\centi\metre}) as a basis. 
			The page headers and footers are centered; the footer contains the current page number. The header either contains the current section name (if a section exists; see e.g. page \pageref{sec:general-functionality-overview}) 
			or the current chapter name).
			\newline Captions include in small, bold font the type of float, its number and are separated by a colon from the description 
			in normal, small font (see e.g. table \ref{tab:example:mhchem}).
		
		\subsection{Language \& Locale}
			The default document language is set to \verb|english| \textcolor{red}{and currently cannot be changed by the user}. 
			The settings for typesetting units (see \mbox{section \ref{sec:units}}) are directly derived from the default language.
		
		\subsection{Font, Visuals \& Color}
			The typesetting design is more classic-minimalist: 
			The user has the choice between the standard serif \verb|lmodern| \LaTeX~font or a sans serif variant as well as 
			three visual styles (\verb|plain|, \verb|fancy-grayscale|, \verb|fancy-colorful|), which differ in highlighting/usage of color.
		
		\subsection{Hyperlinks \& pdf Metadata}
			Per default, hidden hyperlinks are activated. The pdf Metadata is created from the user information: 
			It can be configured via the key value pairs offered by \verb|pdfmetadata|. If there is no additional configuartion provided by the user,
			default values and logic will be used.

	\newpage
	\section{General Functionality Overview}
		\label{sec:general-functionality-overview}
		\subsection{Quotation Marks}
			To facilitate the usage of quotation marks, \verb|minimalthesis| loads the \verb|csquotes| package, which allows to automatically selects the appropriate quotation marks based on language preset and sentence parameters by using \verb|\enquote{text}|.
			\newline For example,\\[0.125cm]
			\begin{tabular}{lcl}
				\verb|\enquote{A quotation mark test.}|& $\rightarrow$ & \enquote{A quotation mark test.}\\
				\verb|\enquote{A \enquote{quotation mark} test.}|& $\rightarrow$ & \enquote{A \enquote{quotation mark} test.}\\
			\end{tabular}
	